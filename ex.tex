\documentclass[12pt,openright]{team}
 \setdefaultlanguage[calendar=gregorian,locale=algeria]{arabic}
\newfontfamily\arabicfont[Script=Arabic,Scale=1]{Sakkal Majalla} 
\newfontfamily\arabicfontsf[Script=Arabic,Scale=1.3]{Aljazeera}
\newfontfamily\arabicfonttt[Script=Arabic,Scale=1.4]{Aljazeera}
% 

 \def\lm#1{\lim\limits_{x\to #1}}
\begin{document}
\begin{center}
 {%
 { \RL{\tt \large
 سلسلة تمارين حول 
الدوال الناطقة} 
}%
}
\end{center}
  
 \exo{
  $\rm{(I}$
  نعتبر الدالة $g$ المعرفة على 
  $\rr$
  كما يلي:
  $g(x)=x^3+3x^2+2$.
  \begin{enumerate}
  \item
  \begin{enumerate}
  \item
  اُدرس تغيرات الدالة $g$.
  \item
  بيّن أنّ المعادلة 
  $g(x)=0$
  تقبل حلًا وحيدًا 
  $\alpha$ 
  في المجال
  $[-4;-3]$.
  \item
  عيّن العدد الطبيعي $n$ بحيث يكون
  $-\dfrac{n+1}{10}< \alpha < -\dfrac{n}{10}$.
  \item
  عيّن إشارة 
  $g(x)$
  حسب قيم $x$.
  \end{enumerate}
  \item
بيّن أنّ:
$\alpha^3=-3\alpha^2-2$.
  \end{enumerate}
$\rm{(II}$
نعتبر الدالة $f$ المعرفة على 
$\ra{-1}$
بــ:
$f(x)=\dfrac{x^3-1}{x^2+2x+1}$.\\
$\left(C_f\right)$
المنحنى الممثل للدالة $f$ في معلم متعامد ومتجانس 
$\oi$.
\begin{enumerate}
\item
\begin{enumerate}
\item
عيّن نهاية الدالة $f$ عند 
$-1$
،ثم فسّر بيانيًا النتيجة.
\item
اُحسب 
$\lim\limits_{x\to -\infty}f(x)$
وَ
$\lim\limits_{x\to +\infty}f(x)$.
\end{enumerate}
\item
\begin{enumerate}
\item
اُحسب 
$f'(x)$
،ثم بيّن أنه من أجل كل $x$ من 
$\ra{-1}$: \\
$f'(x)=\dfrac{g(x)}{(x+1)^3}$.
\item
شكّل جدول تغيرات الدالة $f$.
\end{enumerate}
\item
بيّن أن
$f(\alpha)=\dfrac{-3}{\alpha+1}$
ثم استنتج حصرًا للعدد
$f(\alpha)$.
\item
\begin{enumerate}
\item
بيّن أنّ:
$\lim\limits_{x\to \pm\infty}\left[f(x)-x \right]=-2$
ثم إستنتج معادلة للمستقيم
$(\Delta)$
المقارب المائل للمنحنى 
$\left(C_f\right)$.
\item
اُدرس وضعية 
$\left(C_f\right)$
بالنسبة إلى
$(\Delta)$.
\end{enumerate}
\item
بيّن أنّ للمنحنى 
$\left(C_f\right)$
مماسًا
$(T)$
موازيًا للمستقيم
$(\Delta)$
عند نقطة يُطلب تعيين إحداثياها.
\item
اُكتب معادلة المماس 
$(T)$.
\item
أنشئ 
$\left(C_f\right)$
،
$(T)$
والمستقيمات المقاربة.
\item
عيّن قيم الوسيط الحقيقي $m$ حتى تقبل المعادلة :\\ 
$(m+2)(x+1)^2-3x-1=0$
حلّين مختلفين في الإشارة.
\end{enumerate}
 }
 
 \exo{
$\rm{(I}$
  نعتبر الدالة $g$ المعرفة على 
  $\rr$
  كما يلي:
  $g(x)=x^3-3x-4$.
  \begin{enumerate}
  \item
  اُدرس تغيرات الدالة $g$.
  \item
  بيّن أنّ المعادلة 
  $g(x)=0$
  تقبل حلًا وحيدًا 
  $\alpha$
 حيث 
  $ 2\leq \alpha   \leq 2.25$.
  \item
  عيّن إشارة 
  $g(x)$
  حسب قيم $x$.
  \end{enumerate}
  %
  $\rm{(II}$
نعتبر الدالة $f$ المعرفة على 
$\ra{-1;1}$
بــ:
$f(x)=\dfrac{x^3+x^2+1}{x^2-1}$.\\
$\left(C_f\right)$
المنحنى الممثل للدالة $f$ في معلم متعامد ومتجانس 
$\oi$.
\begin{enumerate}
\item
اُحسب نهايات الدالة $f$ عند حدود مجموعة التعريف.
\item
برهن أنه من أجل كل عدد حقيقي $x$ من 
$\ra{-1;1}$ 
لدينا:\\
$f'(x)=\dfrac{x\, g(x)}{(x^2-1)^2}$.
\item
اُدرس تغيرات الدالة $f$ وشكل جدول تغيراتها.
\item
برهن أن المستقيم 
$(\Delta)$
ذو المعادلة 
$y=x+1$
مستقيم مقارب مائل لـ
$\left(C_f\right)$
بجوار 
$+\infty$
و
$-\infty$.
\item
اُدرس الوضع النسبي بين 
$\left(C_f\right)$
و
$(\Delta)$.
\item
بيّن أن:
$f(\alpha)=1+\dfrac{3\alpha+6}{\alpha^2-1}$
ثم استنتج حصرًا لـ
$f(\alpha)$.
\item
بيّن أن المعادلة 
$f(x)=0$
تقبل حلا وحيدا 
$\alpha'$
حيث
$-1.5 \leq \alpha' \leq -1.25$.
\item
اُرسم 
$\left(C_f\right)$
و
$(\Delta)$.
\item
 $k$
 دالة معرفة على 
 $\ra{\alpha'}$
 بـ:
 $k(x)=\dfrac{1}{f(x)}$.
\item[•]
اُدرس تغيرات الدالة $k$ ثم اُرسم منحناها البياني. 
\item 
ناقش بيانيا وحسب قيم الوسيط 
$m$ 
حلول المعادلة: 
$f(x)=|m-1|$.
 \end{enumerate}
}
\newpage
 \exo{$\rm{(I}$
نعتبر الدالة $g$ المعرفة على 
$\mathbb{R}$
كما يلي :
 $g(x)=x^3-x^2+3x+1$. 
\begin{enumerate} 
\item
اُدرس تغيرات الدالة $g$
.
\item
بيّن أن المعادلة 
$g(x)=0$
تقبل حلًا وحيدًا 
$\alpha$ 
في المجال
$\left[ -1;0\right]$
.
\item
أعط حصرًا لـ
$\alpha$
سعته 
$10^{-2}$
.
\item
عيّن اِشارة
$g(x)$
حسب قيم $x$
.
\item
بيّن أن :
\,
$\alpha^2=-3-\dfrac{4}{\alpha-1}$
.
\end{enumerate}
$\rm{(II}$
نعتبر الدالة $f$ المعرفة على 
$\mathbb{R}$
بــ:
$f(x)=\dfrac{x^3+x-2}{x^2+1}$.
$\left(C_f\right)$
المنحنى الممثل للدالة $f$ في معلم متعامد ومتجانس 
$\left( O;\vec{i};\vec{j}\right)$.
\begin{enumerate}
\item
بيّن أنه من أجل كل  $x$ من $\mathbb{R}$
:  
 $f'(x)=\dfrac{(x+1)g(x)}{(x^2+1)^2}$. 
\item[$\triangleleft$]
اُدرس تغيرات الدالة $f$ .
\item
بيّن أن :
$f(\alpha)=\alpha-\dfrac{2}{\alpha^2+1}$
.
\item[$\triangleleft$]
 استنتج حصرًا للعدد
$f(\alpha)$
باستعمال العدد
$\alpha$
.
\item
عيّن الأعداد 
$a$
،
$b$
،
$c$
و
$d$
بحيث من أجل كل 
$x\in \mathbb{R}$:\\
$f(x)=ax+b+\dfrac{cx+d}{x^2+1}$.
\begin{enumerate}
\item
بيّن أنّ للمنحنى 
$\left(C_f\right)$
مستقيمًا مقاربًا مائلا
$(\Delta)$
يطلب تعيين معادلة له .
\item 
اُدرس وضعية 
 $\left(C_f\right)$
 بالنسبة إلى المستقيم 
$(\Delta)$.
\end{enumerate}
\item
تحقق أنه من أجل كل $x$ من 
$\mathbb{R}$ 
: 
 $f{"}(x)= \dfrac{4(-3x^2+1)}{(x^2+1)^3}$
\item[$\triangleleft$]
بيّن أن للمنحنى
 $\left(C_f\right)$
 نقطتي انعطاف يُطلب تعيين إحداثياتيهما.
 \item
 اُكتب معادلة للمماس 
 $(T)$
 للمنحنى 
  $\left(C_f\right)$
الممثل للدالة $f$
عند النقطة التي 
فاصلتها 
$0$
.
\item[$\triangleleft$]
ماذا تستنتج بالنسبة للمماس 
 $(T)$
 والمستقيم
 $(\Delta)$ 
 ؟
 \item
 أنشئ 
  $\left(C_f\right)$
  ،
  $(T)$
  ،
$(\Delta)$ 
  و المستقيمات المقاربة.
  \item
  ناقش بيانيا حسب قيم الوسيط الحقيقي $m$
  عدد واشارة حلول المعادلة: 
  $m(x^2+1)+2=0$
  .
  \end{enumerate}}
 % \newpage
   \exo{
  $\rm{(I}$
  نعتبر الدالة $g$ المعرفة على 
  $\rr$
  كما يلي:
  $g(x)=x^3-3x-4$.
  \begin{enumerate}
  \item
 اُحسب 
 $\lim\limits_{x\to +\infty} g(x)$
 و
  $\lim\limits_{x\to -\infty} g(x)$.
  \item
  اُدرس اتجاه تغير الدالة $g$ ثم شكّل جدول تغيراتها.
  \item
  بيّن أنّ المعادلة 
  $g(x)=0$
  تقبل حلًا وحيدًا 
  $\alpha$ 
 حيث
  $2<\alpha<2.5 $.
 \item
  عيّن إشارة 
  $g(x)$
  حسب قيم $x$.
  \end{enumerate} 
  $\rm{(II}$
نعتبر الدالة $f$ المعرفة على 
$\ra{0}$
بــ:
$f(x)=\dfrac{x^3+3x+2}{x^2 }$. 
$\left(C_f\right)$
المنحنى الممثل للدالة $f$ في المستوي المنسوب إلى المعلم المتعامد والمتجانس 
$\oi$.
\begin{enumerate}
\item
\begin{enumerate}
\item
  تحقق بأنه من أجل كل $x$ من
  $\ra{0}$:
  $f(x)=x+\dfrac{3}{x}+\dfrac{2}{x^2}$.
  \item
  اُحسب :
  $\lim\limits_{x\to +\infty} f(x)$
  ،
  $\lim\limits_{x\to -\infty} f(x)$
  و
  $\lim\limits_{x\to 0} f(x)$
  ثم فسّر النتائج هندسيًا.
\end{enumerate}
\item
\begin{enumerate}
\item
بيّن أنه من أجل كل $x$ من
  $\ra{0}$:
  $f'(x)=\dfrac{g(x)}{x^3}$. 
\item
استنتج اتجاه تغير الدالة $f$ ثم شكّل جدول تغيراتها.
\item
بيّن أن:
$f(\alpha)=\dfrac{6}{\alpha}+\dfrac{6}{\alpha^2}$
واستنتج حصرًا للعدد
$f(\alpha)$
\end{enumerate}
\item
عيّن النقطة من 
$\left(C_f\right)$
التي يكون فيها المماس 
$(T)$
فيها موازيًا للمستقيم ذو المعادلة 
$y=x$
ثم اُكتب معادلة لهذا المماس.
\item
\begin{enumerate}
\item
بيّن أن المنحنى 
$\left(C_f\right)$
يقبل مستقيمًا مقاربا مائلا 
$(\Delta)$
يُطلب تعيينه.
\item
اُدرس الوضع النسبي لـ
$\left(C_f\right)$
و
$(\Delta)$.
\item
أنشئ 
$(T)$
،
$(\Delta)$
و
$\left(C_f\right)$.
\end{enumerate}
\item
ناقش بيانيًا وحسب قيم الوسيط $m$ عدد حلول المعادلة:
$\dfrac{-mx^2+3x+2}{x^2}=0$.
\end{enumerate}
  }

\end{document}
